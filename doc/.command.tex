

% \PrintMember recibe como argumentos 
%		1) Apellido, Nombre
%		2) Número de Padrón
%		3) E-Mail
%% Se invoca como \PutMember{#1}{#2}{#3}
\providecommand{\PutMember}[3]
{	{#1}	\>\>- \#{#2}\\
		\>{\footnotesize{$<${#3}$>$}}\\
}

\providecommand{\MembersHeader}
{ INTEGRANTES:\hspace{-1cm}\=\+\hspace{1cm}\=\hspace{6cm}\=\\}

\providecommand{\MembersHeaderWithGroup}[1]
{ INTEGRANTES:\hspace{-1cm}\=\+\hspace{1cm}\=\hspace{3cm} \hspace{2cm}\=\\}

% Controla el \pagebreak de la carátula: Configurable con \Pagebreaktrue y \Pagebreakfalse
\newif\ifPagebreak
\newif\ifIndex
\newif\ifSiunitx	% Defino la condición "ifsiunitx"
\newif\ifListings	% Defino la condición "ifListings"
\newif\ifKeywords	% Si en el abstract ponemos palabras clave
\newif\ifPutgroup	% Si en el header aparece el grupo
\newif\ifTodonotes
\newif\ifVideo	
