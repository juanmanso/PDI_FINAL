
En el presente trabajo se analizan dos métodos para la mejora de imágenes
al variar el contraste de las mismas. En clase vimos los métodos de
\emph{streching} y la \emph{ecualización del histograma} o \emph{HE}. El primer
método busca aumentar el rango dinámico del histograma, mientras que el segundo
busca distribuir los valores del histograma (estimación de la función densidad
$f(x)$) de forma tal que la distribución resultante se asemeje a la de una
uniforme. \\

Según el método propuesto por Srinivasan y Balram\footnote{Srinivasan, S
\& Balram, Nikhil. (2006). Adaptive contrast enhancement using local region
stretching.}, la mejora de contraste puede realizarse con ecualizaciones
locales a regiones particulares del histograma ponderadas por una función de
pesos. Esta propuesta (\emph{adaptative contrast enhancement} o \emph{ACE})
muestra ser una mejora a la ecualización del histograma dado que se pueden
obtener resultados que se adapten a la imagen de entrada, resaltando zonas
oscuras o también mejorando el contraste entre objetos.

A continuación se realizará la comparación entre el método \emph{ACE} y el
método \emph{HE}.
