\input{.command.tex}
% En el siguiente archivo se configuran las variables del trabajo práctico
%% \providecommand es similar a \newcommnad, salvo que el primero ante un
%% conflicto en la compilación, es ignorado.
% Al comienzo de un TP se debe modificar los argumentos de los comandos

\providecommand{\myTitle}{TRABAJO FINAL}
\providecommand{\mySubtitle}{Comparación entre HE y ACE}

\providecommand{\mySubject}{Procesamiento de imágenes (86.56)}
\providecommand{\myKeywords}{UBA, Ingeniería, ace, he, histogram, image}

\providecommand{\myAuthorSurname}{Manso}
\providecommand{\myTimePeriod}{Año 2019 - 2\textsuperscript{ndo} Cuatrimestre}

% No es necesario modificar este %%%%%%%%%%%%%%
\providecommand{\myHeaderLogo}{header_fiuba}
%%%%%%%%%%%%%%%%%%%%%%%%%%%%%%%%%%%%%%%%%%%%%%%%

% Si se utilizan listings, definir el lenguaje aquí
\providecommand{\myLanguage}{matlab}
% Crear los integrantes del TP con el comando \PutMember donde
%%		1) Apellido, Nombre
%%		2) Número de Padrón
%%		3) E-Mail
\providecommand{\MembersOnCover}[0]
{
		\PutMember{Manso, Juan} {96133} {juanmanso@gmail.com}
}

\providecommand{\myGroupNumber}{02}


\Pagebreakfalse		% Setea si hay un salto de página en la carátula
\Indexfalse
\Siunitxtrue			% Si quiero utilizar el paquete, \siunixtrue. Si no \siunixfalse
\Todonotesfalse		% Habilita/Deshabilita las To-Do Notes y las funciones \unsure, \change, \info, \improvement y \thiswillnotshow.
\Listingstrue
\Keywordsfalse
\Putgroupfalse		% Habilita/Deshabilita el \myGroup en los headers
\Videofalse
